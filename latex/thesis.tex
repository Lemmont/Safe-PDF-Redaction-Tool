\documentclass[twoside]{uva-inf-bachelor-thesis}
%\usepackage[dutch]{babel}

% Filling your thesis with only lorem ipsum is not advised.
\usepackage{lipsum}
\usepackage{subcaption}
\usepackage{graphicx}
\usepackage{xcolor}
\usepackage{pdfpages}
\usepackage{listings}

\definecolor{mGreen}{rgb}{0,0.6,0}
\definecolor{mGray}{rgb}{0.5,0.5,0.5}
\definecolor{mPurple}{rgb}{0.58,0,0.82}
\definecolor{backgroundColour}{rgb}{0.95,0.95,0.92}

\lstdefinestyle{CStyle}{
    backgroundcolor=\color{backgroundColour},   
    commentstyle=\color{mGreen},
    keywordstyle=\color{magenta},
    numberstyle=\tiny\color{mGray},
    stringstyle=\color{mPurple},
    basicstyle=\footnotesize,
    breakatwhitespace=false,         
    breaklines=true,                 
    captionpos=b,                    
    keepspaces=true,                 
    numbers=left,                    
    numbersep=2pt,                  
    showspaces=false,                
    showstringspaces=false,
    showtabs=false,                  
    tabsize=1,
    language=Python,
}

% Title Page
\title{OpenTRT: Safe Text Redaction In PDF Documents}
\author{Lennaert Feijtes}
\email{contact@ludof.nl - lennaert.feijtes@student.uva.nl}
\supervisors{Maarten Marx}
\signedby{Signees}

\begin{document}
\maketitle

\begin{abstract}

This research examines how text redaction in PDF documents can be accomplished while effectively addressing security concerns and preserving accessibility of the document and non-redacted text. Our findings reveal that a method can be created which both ensures sufficient confidentiality of sensitive information and the preservation of the non-redacted text. Using a sample of both custom and real world PDF documents, we have tested our method on adhering to the safety concerns and possible breaches in accessibility. Our findings reveal that while complete confidentiality can not be ensured without damaging the integrity of the document, no text is being leaked using our method and redacted text is removed from the document. The results highlight that a tool for safe text redaction is possible and that information can be kept sufficiently confidential while maintaining the integrity of the document. This study contributes to the text redaction landscape for publicly made available documents by government bodies and other organisations.
\end{abstract}

\tableofcontents

% https://practicumav.nl/schrijven/schrijfstijl.html

\section{Introduction}
\label{sec:introduction}
 Describe here the part of the field in which this research takes place. It should be clear
from this section how the research question, which will be described later, is positioned within
computer science.
\chapter{Related work}

\cite{bland2022story} presents a method for breaking PDF text redaction using glyph positions; non-redacted character positioning information. In particular, subpixel-sized horizontal shifts in both the redacted and non-redacted characters can be recovered and used to effectively deredact redacted text. Breaking text-redaction schemes relies on knowing the positional information of characters on the page, something that be considered when redacting directly in a PDF instead of scanning and then OCR-ing documents. The paper presents a redaction method where positional information is removed and the document is rewritten in a monospace font. While it ensures safety, accessibility and the non-redacted text are greatly changed or damaged. 
\\\\
Redaction through mosaicing and blurring where text is significantly distorted and transformed such that it is unrecognizable for the eye are popular, especially for images shared through social media. Mosaicing (pixelization) and blurring have been proven to not be viable techniques for text redaction (figure \ref{fig:blurredredaction}). Due to particular regularities in text, enough information may remain to narrow down the possibilities or even to recover the redacted text \cite{hill2016effectiveness}. A simple but powerful class of statistical models, Hidden Markov Models, can be used to recover both short and indefinitely long instances of redacted text. HMMs are used to reover sequences of characters from images of redacted text. 
\begin{figure}[h]
\includegraphics[width=0.7\textwidth]{latex/media/blurredRedaction.png}
\centering
\caption{Example of how a blurred image is first mosaiced, and then the text is inferred by an HMM. Not a viable option for text redaction in documents.}
\label{fig:blurredredaction}
\end{figure}\\
Several methods and tools have been created to automatically find different types of redacted text in PDF documents. The X-Ray Bad Redaction Detector (X-RAY) is a fast and robust tool to identify bad redactions in PDF files \cite{Xray2021}. X-RAY detects trivial redactions where text is badly redacted under a rectangle. A technique using a combination of OCR and morphological operations has been created to detect a wide variety of different types of redaction blocks \cite{van2023detection}.
\\\\
There are several (commercially) available redaction tools which are used for text redaction and anonymization. Octobox \cite{OCTOBOX} and Zylab \cite{Zylab} are two tools frequently used by governmental bodies and municipalities in The Netherlands. \textbf{more about this}

\chapter{Design}

\section{Portable Document Format}
In the context of text redaction, it is essential to consider the relevant components that make up a PDF document and influence security for direct redaction in the document. We consider the PDF document type which contains text data for both the font and the layout of each character (glyph) on a page. 

\subsection{Structure}
A PDF document can be split into 4 distinct parts: 
\\\\
\textbf{Header}. The header of a PDF document serves as its starting point. It contains the critical information about the file which is essential for identifying the PDF format and ensuring compatibility with PDF readers. \\
\textbf{Trailer}. The trailer is found at the end of the document and provides essential information for reading and processing the document. It includes the number of entries in the cross-reference table. It references to the root object of the document's catalog, which contains information about the document's structure, outlines and elements. Finally it may include additional information about encryption, digital signatures and metadta if present. \\
\textbf{Cross-Reference Table(Xref)}. The Cross-Reference Table maintains a record of the location and structure of all objects within the PDF. The Xref table enables efficient random access and editing of the document. It lists the objects, their byte offsets within the file and information about whether an object is in use of has been deleted. \\
\textbf{Body}. The body is where the actual content resides. It includes everything that is visible within the document. This includes text, images, graphical elements, forms and more. The content of a PDF is organized into objects. 
\\\\
The actual contents of a page are often embedded through a stream. A stream may be contained in an object and consists of a sequence of operators with operands which may refer to other objects or information elsewhere.
\begin{figure}[h]
\includegraphics[width=0.85\textwidth]{latex/media/TJexample.png}
\centering
\caption{The TJ text showing operator specifies the glyphs to render, along with their widths and associated positional adjustments by the reference to a font object (not shown here). Adjustments are given in text space units. }
\label{fig:tjexample}
\end{figure}
\subsection{Text rendering}
PDF documents can render text in a wide range of ways, including by the use of a text showing operator such as TJ or Tj. The TJ operator takes as arguments a string of text and a vector of positional adjustments which displace the character with respect to its default position. This position is often determined by the previous character on the line, consisting of a fixed offset that is equivalent to the \textit{advance width} of the previous character. Figure \ref{fig:tjexample} is an example of a TJ text showing operator in practice.
\\\\
Glyph advance widths and glyph shifts create a security concern. The exact width of a redaction and any non-redacted glyph shifts conditioned on redacted glyphs may be used to eliminate potential redacted texts. \cite{bland2022story}. Documents may or may not have these positional adjustments based on the \textit{workflow} that has been used. When saving an email or using 'Save as PDF' in Microsoft Word for example, the documents have \textit{dependent} shifting schemes, while other workflows may produce documents which are \textit{unadjusted}.

\subsubsection{Redaction width}


\subsection{Metadata}
A PDF document may contain extra 'hidden' information which is contained within the metadata of the document. This metadata may contain information about the actual textual contents of the document (figure \ref{fig:metadataexmp}). An example are annotations by authors or reviewers which may comment on a specific part of the document, referring or quoting text which has been redacted from the page, but not from the document. 
\begin{figure}[h]
    \includegraphics[width=0.5\linewidth]{latex/media/metadata.png}
    \centering
    \caption{An example of metadata which may be present in a PDF document. Information such as the author name, subject and description of the document can be retrieved if not also removed.}
    \label{fig:metadataexmp}
\end{figure}\\
The table of contents, embedded files, XML metadata and internal links are also hidden to the eye but may contain sensitive information even after redaction. This information may include author names, document descriptions, links or other references to sensitive personal information which has not been correctly removed.

\section{A new redaction method?}
The process of safely redacting information from a document consists of multiple steps. To redact text, it has to be removed from the PDF by manipulating the content stream of a page.
Parts of a content stream have to be removed or updated, including both string values and positional information about the text. Furthermore, the positional information of the other words on the line have to be adjusted to reduce the leaked information. This is a difficult task because words may be broken up in different text rendering operations, text may be split up in multiple content streams and determining the actual position of a to-be-redacted word can be complicated depending on the workflow used. 
\\\\
For small words
\chapter{Implementation}

\section{to-be-added-name-of-tool}
I present the \textbf{to-be-added-name} redaction method/tool which enables safe text redaction in PDF documents which removes the to-be-redacted text from the document, replaces it with a placeholder text, manipulates the positional information and shifts text to remove white spaces if applicable. I have built my tool/method upon the work of the \textit{PyMuPDF} \textbf{(add link)} Python library which supports text redaction in a wide variety of (PDF) documents. For manipulating positonal information and removing white spaces I have created my own method of reading, interpreting and manipulating content streams which is also built upon PyMuPDF functionality. \textbf{more about the underlying infrastructure...}

\subsection{PyMuPDF}
Building upon the existing library has enabled me to support more documents, focus on the redaction and make the process easier. The library had certain flaws related to text selection, bounding boxes and positional information which I have adapted or solved to fit my own method. 

%\subsubsection{Bounding boxes}
%PyMuPDF makes use of bounding boxes to select text on a page in the document. The standard method for text %selection creates a bounding box which goes outside... \textbf{more}

%\subsubsection{Positional information after removal}
%The redaction method of PyMuPDF is precise and can remove text based upon the bounding box that is given. This %allows for redaction of whole paragraphs, words and even characters.  \textbf{more}

\section{Algorithm for Text Redaction}
The algorithm for text redaction consist of four main steps, with the third step consisting of multiple other steps. 

\subsection{Select redaction}
\textbf{about how text can be selected through user input?}
\\\\
For each page, text is extracted using the \textit{get\_text} method of PyMuPDF. All words and text blocks are extracted and a mapping between the two is made: in which block does a word belong. This is necessary for step 3 where it has to be checked if a word is in the same text block as a redacted word. 
\\\\
Each word and/or selected piece of text is defined by a rectangle; a bounding box. When applying a redaction, an annotation is made with the exact dimension of the defined rectangle. PyMuPDF natively defines a bounding box which tends to be bigger than the exact dimensions of the word. Words are deleted based on intersection with a redaction annotation and so with bigger bounding boxes it may occur that other words or characters are deleted. To solve this issue, the redaction annotation is made smaller. 
\\\\
\textbf{image with the algorithm?}

\subsection{Apply redactions}
When redactions have been selected in the document they are applied \textbf{(more about this...)} and a replacement text is inserted. The style of the replacement text is based on the original text. Depending on the original text, the rectangle of the to-be-inserted, which is initially is the same as the original, text may have to be adapted to fit the new text. 
\subsection{Positional adjustments}
\textbf{talk about how relevant text is selected based on y positions and text block}
\\\\
\textbf{pymupdf lines/content stream reading and interpreting}
\\\\
\textbf{applying adjustments}
\\\\
\textbf{moving text}
\subsection{Metadata}

\section{Limitations}
\textbf{Types of documents and why (complexity) limited functionality}
\textbf{About positional information and why chosen}
\chapter{Evaluation}


\section{Validation}
Effective redaction should adhere to two principles: what has to be removed has to be really removed from the document and the rest of the document should be kept as it is. To validate my method, input has to be compared to the output; is the text that should be removed gone and only that part of the document? \textbf{to be added: xray-tool}

\section{Text corpus}
For the evaluation of my method I simulate redaction on documents from multiple sources. First is my own set of documents which contains varying cases. From a couple of lines of text to multiple pages, and pages including headers to a page with table of contents. Multiple fontsizes, two columns and annotations are all included. Secondly I use a part of the 1 million page corpus of Dutch Freedom of Information Act(Woo) documents. This part consists of \textbf{amount} decision letters coming from mostly dutch municpialities, provicnes and \textit{waterschappen} written between \textbf{year 1-year 2} by legally trained civil servants. The machine readable text is extracted from these documents using \textit{PyMuPDF}.

\section{Experiment Procedure}
To validate the redaction method, we set up an experiment which takes as input a set of PDF documents, runs a validation experiment and returns if this has been successfully or not. This procedure is characterized by the following steps:
\begin{enumerate}
    \item Select a document for redaction.
    \item Depending on the document, select a string value to be redacted. This can be either a custom string value, for example \textbf{M. Jackson}, or a number of random words per page. 
    \item Run the algorithm on the document given the (randomly) selected text. This outputs a result PDF file.
    \item Extract the text from both the original and newly redacted PDF file.
    \item Loop over the text of the original file. Count the occurrences of each word.
    \item Loop over the text of the new file. Subtract the occurrences of each word from the previously counted words of the original file. If redaction went right, there should be occurrences of a replacement value such as \textbf{[x]}. These are skipped.
    \item Finally, subtract the redacted text items from the counted words of the initial document. Redaction is valid if no occurrences are left.
\end{enumerate}

In addition to text in the PDF, I have validated if to-be-redacted text values are present in the metadata or annotations (if there are any). \textbf{to be added: X-ray tool}


\chapter{Results}

\section{Experiment results}
\subsection{Text redaction}


\begin{table}[h]

\caption{The results of the redaction experiment where one randomly selected character sequence per page is labelled for redaction. Horizontally, the validation criteria are placed together with the amount of documents and the the document type. Vertically, the results for each category of documents (separated by type) are displayed.}
\hspace{-25mm}
\begin{tabular}{llcccccl}
 & \multicolumn{5}{c}{Redaction of 1 randomly selected character sequence per page} &  \\ \cline{2-7}
 & \multicolumn{1}{c}{} & \multicolumn{4}{c}{validation criteria} &  \\ \cline{2-7}
 & \multicolumn{1}{c}{type} & \begin{tabular}[c]{@{}c@{}}Text has been\\ removed\end{tabular} & \begin{tabular}[c]{@{}c@{}}Positional adjustments \\ have been\\ manipulated\end{tabular} & \begin{tabular}[c]{@{}c@{}}Non-redacted text \\ roughly same position\\ as original document\end{tabular} & \begin{tabular}[c]{@{}c@{}}Non-redacted words are\\ easily readable\end{tabular} & \begin{tabular}[c]{@{}c@{}}Amount\end{tabular}  \\ \cline{2-7}
 & \multicolumn{1}{l|}{\textbf{Custom}} & 0.96 & $0.92^{*}$ & 0.92 & 1.0 & 27 &  \\
 & \multicolumn{1}{l|}{\textit{simple}} & 1.0 & 1.0 & 1.0 & 1.0 &  8 & \\
 & \multicolumn{1}{l|}{\textit{medium}} & 1.0 & 1.0 & 1.0 & 1.0 &  6 & \\
 & \multicolumn{1}{l|}{\textit{hard}} & $0.86^{a}$ & 0.86 & $0.63^{***}$ & $1.0^{***}$ & 7 & \\
 & \multicolumn{1}{l|}{\textit{edge case}} & $1.0^{*}$ & $0.83^{*}$ & $1.0^{*,***}$ & $1.0^{*,***}$ & 6 & \\
 & \multicolumn{1}{l|}{\textbf{Found}} & 0.85 & $0.71$ & 0.83 & 0.98 & 48 &  \\
 & \multicolumn{1}{l|}{\textit{MS Word}} & $0.70^{a}$ & 0.65 & $0.70^{a}$ & 1.0 & 24 &  \\
 & \multicolumn{1}{l|}{\textit{XEP}} & $0.95^{a}$ & $1.0^{**}$ & 1.0 & 1.0 & 19 &  \\
 & \multicolumn{1}{l|}{\textit{Aspose.Words}} & 1.0 & 0.0 & 1.0 & 1.0 & 2 &  \\
 & \multicolumn{1}{l|}{\textit{Acrobat Distiller}} & 1.0 & 0.0 & 1.0 & 0.5 & 2 &  \\
 & \multicolumn{1}{l|}{\textit{3-Heights(TM)}} & 1.0 & 0.0 & 1.0 & 1.0 & 1 &  \\
 & \multicolumn{1}{l|}{\textbf{All}} & 0.89 & 0.79 & 0.88 & 0.99 & 75 &
\end{tabular}
\label{tab:redaction-res}
\end{table}

\begin{itemize}
    \item *: Includes blank pages where no redaction can be taken place. 
    \item **: XEP PDF files have no positional information to be manipulated. 
    \item ***: Multi-column pages have differing results based on the redacted word's position and length.
    \item a: Documents where an error has occurred during the redaction process have been labelled as false.
\end{itemize}
\newpage
\subsection{Metadata}
\begin{table}[h]
\centering
\caption{The results of the redaction experiment where a given value (Lennaert Feijtes) has been removed from the metadata in the custom documents. Horizontally, the validation criteria together with the amount and type of documents are displayed. Horizontally, the results for each document type are displayed.}
\begin{tabular}{llccc}
 & \multicolumn{4}{c}{Redaction of author name in metadata} \\ \cline{2-5} 
 & \multicolumn{4}{c}{validation criteria} \\ \cline{2-5} 
 & \multicolumn{1}{c}{type} & \begin{tabular}[c]{@{}c@{}}author name \\ has been removed\\ from standard metadata\end{tabular} & \begin{tabular}[c]{@{}c@{}}author name \\ has been removed\\ from XML metadata\end{tabular} & amount \\ \cline{2-5} 
 & \multicolumn{1}{l|}{\textbf{all}} & 1.0 & 1.0 & 27 \\
 & \multicolumn{1}{l|}{\textit{simple}} & 1.0 & 1.0 & 8 \\
 & \multicolumn{1}{l|}{\textit{medium}} & 1.0 & 1.0 & 6 \\
 & \multicolumn{1}{l|}{\textit{hard}} & 1.0 & 1.0 & 7 \\
 & \multicolumn{1}{l|}{\textit{edge case}} & 1.0 & 1.0 & 6
\end{tabular}
\label{tab:metadata-table}
\end{table}

\chapter{Discussion}

\section{Interpreting the results}
The results in table \ref{tab:redaction-res} clearly indicate that the redaction of text is mostly successful while maintaining accessibility for both custom and found PDF documents. Especially the custom documents were found to be easily redactable, with only a few cases of high complexity resulting in an error. These errors were often the result of not being able to correctly detect and/or find operations of text on the same line.  
\\\\
The accessibility was in almost none of the cases greatly damaged, taking into consideration the documents that resulted in a error. An unexpected result was redaction in multi-column page PDF documents (Figure \ref{fig:example-column-page}). Here the non-redacted text in one column was greatly altered by redaction of a word in the other, meaning that multi-column pages are not yet fully supported by OpenTRT. Overall, almost 90\% of the PDF documents closely resembled the original file in terms of visual layout and readability after redaction.
\\\\
Documents that were 'found', consisting of decision letters and other types of documents coming from mostly dutch municipalities, provinces and waterschappen, overall did have text successfully redacted. The rate of successful redaction of PDF documents that were found to be created using Microsoft Word was lower than expected, especially when compared to the custom made documents. A possible explanation for this may be due to the fact that some of these documents might have been edited or redacted using another software after creation before (there were a couple of documents where text seems to have been redacted or edited).
\\\\
In PDF documents that were created using another workflow no positional adjustments have been manipulated. This was as expected because during development it became clear that support for more types of PDF documents was too complex. However, after redaction both the text was removed and accessibility preserved. 
\\\\
Finally, table \ref{tab:metadata-table} clearly indicates that in all custom documents, the value \textit{Lennaert Feijtes} has been successfully removed from the metadata of the document. 


\section{Conclusion}
This research has examined how text redaction in PDF documents can be accomplished while
effectively addressing security concerns and preserving accessibility of the document and non-redacted text. The main goal was to find a middle ground between the these two essential aspects. Our findings have revealed that a method can be created which both ensures sufficient confidentiality of sensitive information and preservation of the non-redacted text by directly manipulating the internal representation of text for most PDF documents that have been created using the 'save as PDF' functionality of Microsoft Word. By Using a
sample of both custom and real world PDF documents that vary in compelxity, we have tested our method on adhering to the safety concerns and possible breaches in accessibility. Our findings reveal that while complete confidentiality can not be ensured for every document and document type without damaging the integrity of the document,
no text is being leaked using our method and redacted text is removed from the document.
The results highlight that a tool for safe text redaction is possible and that information
can be kept sufficiently confidential while maintaining the integrity for most documents. This study contributes to the text redaction landscape for publicly made available documents by government bodies and other organisations.



\subsection{Further recommendations}
Redacting text directly in PDF documents may lead to sufficient confidentiality, but it does not ensure 100\% safety against deredaction. After a PDF document has been made, positional information of all text will be present and will have to be removed to ensure this. However, document integrity and visual aesthetic will be greatly damaged in the process. It is recommended to remove to-be-redacted text from a document before it is converted to a PDF document i.e. replacing text with \textit{REDACTED}. This leaves little to no positional information that can lead to the original value. Furthermore, directly redacting text before creating the PDF will lead to better accessibility. No manipulation after PDF creation is needed and thus there will be no changes to the internal structure and style. 
\\\\
For any hidden information, preventive measures can be taken to reduce and minimize the amount of information leaked. Manually setting metadata fields through a document editor leads to both better confidentiality and accessibility of the document by providing only the necessary information about the document. Furthermore, manually removing any unnecessary file attachments, (hidden/covered/transparent) text, table of contents entries, links, bookmarks and annotations before converting a document to PDF results in fewer possible breaches in confidentiality.

\section{Ethical considerations}
This research field aims at protecting sensitive information from PDF documents which may be considered as something good. Extensive knowledge about positional information and how positional adjustments of text are dependent on each other can both be a good or bad thing, depending on the goal of the party involved. This information can both be used to create new solutions for safe redaction or deredact already existing documents and cases where these vulnerabilities are exposed. 

\section{Further work}
Building upon already available methods, like PyMuPDF, and expanding upon them by creating a way of manipulating positional adjustments and removing any confidential information from non-visual content for a wide(r) range of PDF types would be a logical next step in order to create a method/tool that safely redacts information from PDF documents. While problems can be prevented for new documents, for already existing PDF documents it will be necessary to have a safe redaction tool.
\\\\
Furthermore, improving on the implementation and results in this paper is also a logical next step. Especially, dissecting the internal structure of the PDF and how text is rendered. A better alternative to extract operations and map them to the same line as a redaction may improve effective redaction. 
\\\\
Expanding further upon the possibilities of positional information manipulation will allow for more confidentiality while maintaining accessibility and visual aesthetic. Reconstructing different positioning schemes (i.e. MS word schemes) will allow for better understanding of positional adjustments and the creation of better defenses against deredaction.

\chapter{Acknowledgements}
I would like to thank my supervisor, Maarten Marx, for their guidance, enthusiasm and advice during this thesis project. My supervisor provided valuable assistance in steering me towards a more successful and achievable trajectory for this project.
\\\\
I would like to express my sincere appreciation to Maxwell Bland, one of the authors of \textit{Story Beyond the Eye: Glyph Positions Break PDF Text Redaction}, for his valuable explanation and further elaboration on some aspects of his work; especially on the recommendations proposed on adding noise to positional adjustments.
\appendix
\chapter{PDF Documents}
\begin{figure}[h]
    \hspace{-18mm}
    \includegraphics[scale=0.6]{latex/media/exm.png}
    \caption{A PDF document created in Microsoft Word where some random text has been inserted onto the page with a header.}
    \label{fig:example-pdf-lorem-ipsum}
\end{figure}

\begin{figure}[h]
    \hspace{-18mm}
    \includegraphics[scale=0.4]{latex/media/multicol.png}
    \caption{A PDF document created in Microsoft Word where some random text has been inserted onto the page. The page is divided in a two column based layout.}
    \label{fig:example-column-page}
\end{figure}

\chapter{Codebase}
The code for this project, including all code for evaluation and test documents, can be found on the GitHub of this project: link 

% references
\bibliographystyle{ACM-Reference-Format}
\bibliography{bibliographies/references}
\end{document}

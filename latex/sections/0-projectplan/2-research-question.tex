\section{Research Question}
% describe the problem that will be worked on. This often takes the form of a reflection with regard to the “state-of-the-art” just described. As part of the research question, it is also described what the project will deliver: the product that will be delivered at the end, eg the results of a research, the source code of developed software, documentation.

\textbf{Research Question}. Many security concerns within the PDF text redaction field exist and often confidentiality and privacy are compromised as a result of either human error or information left behind in the document. In my research I want to answer the following: \textit{How can we design and develop a PDF text redaction tool that effectively addresses security concerns, including the prevention of information leakage through subpixel-sized horizontal shifts, while preserving non-redacted text.} 

\textbf{The result}. The project will deliver a PDF text redaction method based on research of relevant literature. The redaction method will be tested while designing it using test documents created by myself as well as real-world examples. Relevant tools will be used to assess the safety of the redactions produced by the method in the test cases. \textit{X-Ray Bad Redaction Detector}\cite{Xray2021} will be used to check for non-excising redactions where redacted text is retained in the PDF and only visually obscured and the \textit{Edact-Ray Tool Suite}\cite{bland2022story} to locate both non-excising and excising redactions. For the latter the code is not publicly available, but for scientific purposes and upon request, the author is willing to share the full code and the dataset \cite{MaxellCode2021}. To check metadata, \textit{ExifTool} seems like a promising candidate \cite{ExitfoolHarvey}. Finally libraries such as \textit{pdfid}, \textit{pdf-parser} and \textit{peepdf} seem useful.
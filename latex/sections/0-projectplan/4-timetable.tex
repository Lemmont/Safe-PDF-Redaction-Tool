\section{Project Plan}
\label{sec:project_plan}
% Describe a timeline via a Gantt chart or table with achievements per week.
 Describe how the available time is expected to be spent. It is often difficult to identify
all activities in advance, let alone to indicate accurately on the day what will be completed on which
date. In general, it is possible to identify a number of phases (eg literature research, design,
implementation, experiments, writing a thesis, etc.) and to make a weekly schedule. Also consider
the interim results, the dependencies that may exist and whether there are critical dependencies
that prevent the project from proceeding and what the alternative plan is in that case. A Gantt or
Pert chart is often used for this.

% It is important to plan in buffers and time for actually writing the thesis. Do not underestimate the time for the latter, especially as 10 pages do not seem too much (but they are).

% Timeline
% Great and straight forward but lacks documentation. 
% Interested users can have a look here:
% https://github.com/lwiseman/chronology
% Unfortunately, the last commit on main was in 2015
\begin{center}
    \begin{chronology}[1]{0}{13}{60ex}[\linewidth]
        \event[0.01]{1}{Data from Company}
        \event[1]{2}{Feature Engineering}
        \event[2]{3}{Data Enrichment}
        \event[3]{6}{Setup}
        \event[6]{8}{Training}
        \event[8]{9}{Evaluation}
        \event[9]{12}{Writing}
        \event[12]{13}{Buffer}
    \end{chronology}
\end{center}

\newpage

% Gantt Chart
% For more complex Gantt charts see documentation here: 
% http://mirror.ox.ac.uk/sites/ctan.org/graphics/pgf/contrib/pgfgantt/pgfgantt.pdf
\begin{ganttchart}[
    expand chart=0.9\linewidth,
    vgrid,
    hgrid
    ]{0}{12}
        % Titles
        \gantttitle{Weeks}{13} \\
        \gantttitlelist{1,...,13}{1} \\

        % Group
        \ganttgroup{Data Aggregation}{0}{2} \\  % elem 0
        % Concrete tasks
        \ganttbar{Data from Company}{0}{0} \\  % elem 1
        \ganttbar{Feature Engineering}{1}{1} \\  % elem 2
        \ganttbar{Data Enrichment}{2}{2} \\  % elem 3
        % More groups, further tasks ommitted
        \ganttgroup{Setup}{3}{5} \\  % elem 4
        \ganttgroup{Training}{6}{7} \\  % elem 5
        \ganttgroup{Evaluation}{8}{8} \\  % elem 6
        \ganttmilestone{Finish Experiments}{8} \ganttnewline % elem 7
        \ganttgroup{Writing}{9}{11} \\  % elem 8
        \ganttgroup{Buffer}{12}{12} % elem 9
        
        % Connectors
        \ganttlink{elem0}{elem4}
        \ganttlink{elem4}{elem5}
        \ganttlink{elem5}{elem6}
        \ganttlink{elem6}{elem7}
        \ganttlink{elem7}{elem8}
  \label{ganttchart}
\end{ganttchart}

\section{Introduction}
\label{sec:introduction}
% Describe here the part of the field in which this research takes place. It should be clear from this section how the research question, which will be described later, is positioned within computer science.

Documents may contain sensitive personal information which can not be shared when, for example, communication between two parties are made publicly available. It is important that this information is left out of the document before being shared. Text is often redacted by masking text with black boxes or by blurring which would ensure confidentiality.. However, this is not always the case \cite{failures2019}. 

Text redaction tools are essential in various domains. These domains include legal, healthcare, government and corporate sectors where privacy and data protection are of paramount importance. Text redaction tools are used to safeguard sensitive information by hiding or obliterating portions of text, making it unreadable or unintelligible while preserving the structure of the document. No reference to the redacted text should be left in the document after the redaction and no means of identification should be possible by its relation to the non-redacted text i.e. subpixel-sized horizontal shifts that can be recovered from both the redacted and non-redacted characters. 

This research is situated in the field of text redaction tools. In this field there is a focus on developing and improving methods that enable the secure and efficient redaction of text in digital documents, in this case PDF. Safe text redaction ensures confidentiality and the safeguarding of sensitive personal information, making this research a significant contribution to the field of text redaction.

Overall, this research deals with the intersection of information security, document processing and algorithmic techniques.
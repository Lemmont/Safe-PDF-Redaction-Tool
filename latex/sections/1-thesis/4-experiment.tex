\chapter{Evaluation}


\section{Validation}
An effective redaction implementation should adhere to the principles defined earlier in designing a method that strikes a balance between confidentiality and accessibility. In summary, these principles are:

\begin{enumerate}
    \item The to-be-redacted text has been removed visually.
    \item The to-be-redacted text has been removed non-visually.
    \item Non-redacted text should not be removed visually, but may be removed from the non-visible parts that are not related to the rendering of the visible text (i.e. embedded files).
    \item Non-redacted text may be manipulated but still be readable by both human and machine, while preserving the original style as much as possible.
    \item Edited or inserted text may be moved, but must be kept to a minimum so that the result still resembles the layout of the original file.
\end{enumerate}

Principles 1 and 2 ensure the actual removal of text from both the visual content and hidden information of the document. Principle 3 allows for the removal of any hidden information even if there is no to-be-redacted information present, such as any (unused) file attachments or annotations. Principle 4 permits the manipulation of non-redacted text as long as it remains readable, involving editing positional adjustments. Finally, Principle 5 allows for changing the position of text as long as the result still resembles the original file. This means that text may be shifted horizontally to remove any white spaces created and vertically to align inserted text with non-redacted text. Words should not be moved to a new line or become invisible after manipulation.
\\\\
The to-be-redacted information must be removed from any potential source where hidden information may be present, either through careful examination and deletion or the complete removal of the source. However, parts like metadata or the table of contents may be important for accessibility, making it desirable not to delete them entirely.


\section{Text corpus}
For the evaluation of my method we simulate redaction on documents from multiple sources. First is my own set of \textbf{27} PDF documents which contains cases varying in complexity and layout; from a couple of lines of text to multiple pages, and pages including headers to a page with table of contents. Multiple edge cases are included, such as a blank page, two columns and a page with a small table. These documents were originally created in Microsoft Word and then converted to PDF using the 'Save to PDF' functionality. Secondly we use a part of the 1 million page corpus of Dutch Freedom of Information Act(Woo) documents. This part consists of \textbf{48} decision letters coming from mostly dutch municpialities, provicnes and \textit{waterschappen} written  by legally trained civil servants. The machine readable text is extracted from these documents using \textit{PyMuPDF}.

\section{Experiment Procedure}
To validate the redaction method, we set up an experiment which takes as input a set of PDF documents which have not been redacted yet, performs redaction on one or more (randomly) chosen character sequences per page, runs a set of validation tests on the result and returns if these have been successful or not. This procedure is characterized by the following steps:
\begin{enumerate}
    \item Select a document for redaction.
    \item Choose one or more character sequences (randomly) per page from the document for redaction
    \item Run OpenTRT on the document given the selected redactions. This outputs a resulting redacted PDF file.
    \item Extract the text from both the original, the newly redacted PDF file and the selected redactions.
    \item Loop over the text of the original file in natural reading order, which gives the index \textbf{i}. Keep track of the current redaction that is being checked, index \textbf{j} and do the following:
        \begin{enumerate}
            \item Check: original[i].value == new[i - j].value
            \item If True: continue
            \item If Not: original[i].value == redactions[j].value
                \begin{enumerate}
                    \item If True: j++, continue
                    \item If Not: ERROR not correct 
                \end{enumerate}
        \end{enumerate}
\end{enumerate}

Finally, it is also checked if the positional adjustments for the lines where one or more character sequence(s) have been redacted, have been manipulated.

\subsection{Metadata}

In addition to text in the PDF, I have validated if to-be-redacted text values are present in the (xml) metadata (if there is any). Specifically if my name (\textit{Lennaert Feijtes}) is present in the author field of the metadata and if this has been succesfully removed. Metadata is extracted from both the original and resulting file, its fields split, their values examined and finally compared. This procedure is characterized by the following steps:

\begin{enumerate}
    \item Extract the metadata from both the original and resulting redacted PDF file, and the selected redactions.
    \item Loop over the fields of the metadata in the redacted file, which gives index \textbf{i} and do the following:
        \begin{enumerate}
            \item Check: new[i].value == original[i].value.replace(redactions, "")
            \item If True: continue
            \item If Not: ERROR not correct
        \end{enumerate}
\end{enumerate}

\subsection{Look and feel}
To validate if text has not been manipulated to a point where document integrity has been damaged and the style has deviated to far from the original document, manual examination is required. The following criteria will be checked on:

\begin{enumerate}
    \item Are all words roughly in the same position as in the original, taking into account possible shifts to remove white spaces?
    \item Are all words easily readable? Have individual characters of words or entire words overlapped?
\end{enumerate}

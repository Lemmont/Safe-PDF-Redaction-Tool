\chapter{Related work}
Various tools and services are available for text redaction in PDF documents, including well-known commercial option Adobe Acrobat Pro, but also less known options such as OctoBox \cite{OCTOBOX} and Zylab \cite{Zylab}. These last two tools, often utilized by governmental bodies in the Netherlands, offer redaction as part of a broader document management environment. While these tools claim to ensure safe redaction by removing text, none of them refer to anything about the removal or manipulation of positional information and other hidden information from the document.
\\\\
In certain cases, tools employ techniques that involve 'taking a photo' of a page, redacting words, and subsequently using optical character recognition to detect non-redacted text, making it readable again. Unfortunately, this process often results in reduced accessibility, with unrecognized words, the creation of non-existent words, or in the worst case: only images of the document remaining.
\\\\
\cite{bland2022story} introduces The Edact-Ray Tool Suite, which focuses on locating, analyzing, and protecting redactions by removing the actual textual contents of redactions. This tool optionally eliminates all non-redacted positional adjustments and adjusts the size of spaces between words based on a monospace font. While this approach minimizes information leakage about redacted text and maintains readability for both humans and computers, it may compromise the visual aesthetic and overall document structure. The paper proposes several recommendations to defend against deredaction of redacted text through the manipulation of positional information. One notable suggestion involves adding noise to positional adjustments on a line of text to increase the computational complexity of deredaction while preserving the visual integrity of the document.
 


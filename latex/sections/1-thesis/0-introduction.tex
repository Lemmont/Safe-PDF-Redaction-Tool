\chapter{Introduction}

Redaction is 'the act of removing words or information from a text before it is printed or made available to the public (Cambride Dictionary). Information or text is often redacted when it conveys sensitive information. This can be due to privacy and legal reasons, or because the text reflects the opinion of someone, or because of the risk of commercial conflicts from the publication of data. %An example is Personal Identifiable Information (PII), which is information which can be used to distinguish an individual's identify such as their name, alone, or when combined with other information which is linked or linkable to a specific individual [ref]. However, it is not limited to only personal information, especially in legal documents. %Text redaction is essential in various domains. From the legal to healthcare sectors and from governmental to corporate sectors, privacy and data protection are of paramount importance due to the aforementioned protection of sensitive information and to comply with privacy regulations. Two essential aspects are demanded of effective redaction. First, redaction has to \textit{really} remove sensitive information from a text or document. Secondly, the information that does not have to be redacted has to be kept intact and visible for the reader. Governments and organisations often can not adhere to these two essential aspects of redaction. In many instances, the focus on safety is to the detriment of the non-redacted text which is damaged; more text is redacted or becomes unreadable. In the worst case all text is removed and only images of documents are made publicly available. 
Multiple countries have \textit{Freedom of Information Acts} \cite{USAFia}, that require governmental bodies to release documents upon request of civilians. In The Netherlands the \textit{Wet open overheid (Woo)} \cite{WooWebsite} serves as such a law. This has resulted in different commercial text redaction tools in use by governments to speed up the redaction process. Different types of redacted text exist, from completely black filling (black-boxes) to gray bars to completely white, and even manual crossing out with a pen. 
\\\\
\textbf{image about types of redactions}
\\\\
Two essential aspects are demanded of effective redaction. First, redaction has to \textit{really} remove sensitive information from a text or document. No sensitive information should be left in the rest of the document (metadata etc.). Secondly, the information that does not have to be redacted has to be kept intact and visible for the reader. Governments and organisations often can not adhere to these two essential aspects of redaction. In many instances, the focus on safety is to the detriment of the non-redacted text which is damaged; more text is redacted or becomes unreadable. In the worst case all text is removed and only images of documents are made publicly available, especially when first scanning and then again OCR-ing documents is the predominant technique used by text-redaction tools. This is the case for the Dutch redacted text landscape.
\\\\
I present a safe PDF redaction tool that ensures confidentiality of sensitive information and does not damage non-redacted text using... \textbf{more to be added...}
\\\\
By being able to account for all safety concerns, we are able to test our method using custom and real-world examples. Using tools such as X-ray, we were able to prove if our method was safe or not... \textbf{more to be added...}
\\\\
Our method performs... \textbf{more to be added...}
%The Woo arranges the right to information of everything the government does and enables the citizen to request information about the preparation and the execution of policy of a governing body through so-called Freedom of Information (FIA) requests. The Woo also obliges government organisations to actively publish information to the public. \\

%While the Woo makes publishing of decisions and requested documents mandatory, it does not regulate the working method of Woo-bureaucrats nor the quality of the publications. Publicly published Woo files often exist of scanned in PDF files which are not readable for a computer and without any standardized metadata about the file or documents. \\
%Article 2.4 of the Woo [ref] mandates that governing bodies should take care that documents are in a good, an ordered and an accessible state. When such a body releases information conform this law, it should do this in such a way as to reach the interested citizen. Released documents should be easily findable, accessible, interoperable and reusable (FAIR data principles). 


%Data protection laws such as the \textit{Algemene Verordening Gegevensbescherming} (AVG) in the Netherlands, which is based on the European Union regulation on information privacy which enforces certain rules on the protection of natural persons with regard to the processing of personal data and on the free movement of that data...

%\textbf{UAVG}

%The \textit{Algemene Wet Bestuursrecht} (AWB) is the General Administrative Law Act in the Netherlands which functions as a fundamental legal framework that governs administrative law and the relationship between government authorities and citizens. This law includes provisions related to privacy and data protection, primarily safeguarding the individual when interacting with government authorities.

%\textbf{De Awb bevat regels voor administratieve procedures en besluitvorming door overheidsinstanties, inclusief het verwerken van persoonlijke informatie van ambtenaren en burgers in administratieve documenten.}

%These laws provide a legal basis which ensures that sensitive and PII is appropriately protected in multiple ways. Organizations and government bodies need to identify sensitive or personal information in documents. This includes names, social security numbers, medical records etc. There needs to be a legal basis for processing sensitive data based on consent, legal obligations, the performance of a contract or other legitimate interests. The specific purpose for which the information is processed has to be well defined and in line with the legal basis. Data that is not necessary for this specific purpose should be either removed or obscured; data minimization. Individuals whose data is used have rights under the AVG, including the right to know if their data is being processed, access to their data and to rectify inaccuracies. Document retention, erasure of information, consent and notice, record keeping and accountability are also part of the legal basis.  



